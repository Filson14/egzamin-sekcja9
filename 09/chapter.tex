\chapter{Języki i technologie webowe}
\PartialToc
%\startcontents[chapters]
%\printcontents[chapters]{}{1}{\section*{\contentsname}}
\section{IT1A\_W10,IT1A\_U10}
\textbf{Droga pakietu w sieci Internet pomiędzy dwoma
węzłami, tj. lista adresów węzłów odwiedzanych przez pakiet}

Droga takiego pakietu zależy od stanu tablic routingu. Jeżeli stosowany jest routing dynamiczny wtedy pierwsza próba wysłania pakietu się nie powiedzie - router nie zna adresu mac celu i musi go odnaleźć w związku z czym porzuca pakiet, natomiast jeśli w tablicy routingu znajduje się już adres mac węzła docelowego pakiet pójdzie dalej. Całość oczywiście zależy od stopnia skomplikowania sieci (np. ilość routerów po drodze).
\\
\\
Ścieżka pakietu w prostej sieci jest bardzo fajnie i obrazowo opisana tutaj:\\
http://jaredheinrichs.com/tracing-packet-flow-between-a-2-switches-and-a-router.html


\vspace{0.4cm}
\noindent 


\section{IT1A\_W10,IT1A\_W06, IT1A\_U05, IT1A\_U03, IT1A\_U10 }
\textbf{Jak długi będzie czas wykonania poniższego programu napisanego w języku PHP? Zakłada się, że program uruchamiany jest jako aplikacja WWW tj. dostępny jest pod określonym adresem URI, a interpreter PHP uruchamiany jest przez serwer WWW.}

\begin{lstlisting}[language=php]
1 <? php
2 echo 'start';
3 sleep(6);
4 ?>

\end{lstlisting}

Wywołanie \textbf{sleep(6)} powoduje "uśpienie" programu na 6 sekund. Do tego czasu należy doliczyć także czas wykonania instrukcji \textbf{echo 'start'}, więc ogólny czas wykonania programu będzie większy niż 6 sekund. Sprawdzone za pomocą funkcji \textbf{microtime()}.

\vspace{0.4cm}
\noindent


\section{IT1A\_W10,IT1A\_W06, IT1A\_U05, IT1A\_U03}
\textbf{Zaznacz prawdziwe stwierdzenia dotyczące poniższego kodu w języku JavaScript.}


\begin{lstlisting}
function updateAjax() {
    xmlhttp = new XMLHttpRequest();
    xmlhttp.onreadystatechange = function() {
        if (xmlhttp.readyState == 4 && xmlhttp.status == 200) {
            document.getElementById("stime").innerHTML = xmlhttp.responseText;
        }
    }
    xmlhttp.open("GET", "date.php", true);
    xmlhttp.send();
    window.setTimeout("updateAjax()", 1000);
}
window.setTimeout("updateTime(); updateAjax();", 5000);
\end{lstlisting}

\begin{itemize}
\item{Event \textbf{XMLHttpRequest.onreadystatechange} jest triggerowany za każdym razem, gdy
\textbf{XMLHttpRequest.readyState} się zmienia}

\item{\textbf{XMLHttpRequest.readyState == 4} oznacza, że wysłany przez nas request został zakończony.}

\item{\textbf{XMLHttpRequest.status == 200 } (HTTP status 200) oznacza, że nasz request został przetworzony poprawnie}

\item{\textbf{XMLHttpRequest.responseText} jest odpowiedzią serwera na nasz request. W przypadku powyższego kodu elementowi \textbf{na naszej stronie} o id == 'stime' zostanie przypisana odpowiedź serwera.} 

\item{Funkcja \textbf{XMLHttpRequest.open(method,url,async)} określa typ requestu jaki zostanie wysłany. }

\item{\textbf{window.setTimeout(function, milliseconds)} powoduje pojedyncze uruchomienie funkcji (jak widać w powyższym kodzie może ich być kilka) określonej przez parametr \textbf{function} po odczekaniu czasu określonego przez parametr \textbf{miliseconds}.}

\end{itemize}
Komunikacja AJAX rozpocznie się po 5 sekundach od zinterpretowania kodu. Następnie funkcja \textbf{updateAjax()} będzie wywoływana co jedną sekundę (sama ustawia sobie timeout).


\vspace{0.4cm}
\noindent

\section{IT1A\_W05, IT1A\_U05}
\textbf{Dany jest dokument XML oraz odpowiednie DTD. Zaznacz prawdziwe stwierdzenia.}\\

\textbf{DTD} -  rodzaj dokumentu definiujący formalną strukturę dokumentów XML, HTML, XHTML lub innych
 pochodzących z rodziny SGML lub XML. Definicje DTD mogą być zawarte w pliku dokumentu, którego strukturę definiują, przeważnie jednak zapisane są w osobnym pliku tekstowym, co pozwala na zastosowanie tego samego DTD dla wielu dokumentów.\\

\textbf{XMLSchema} - opracowany przez W3C standard służący do definiowania struktury dokumentu XML. XML Schema stanowi alternatywę dla DTD, przy czym posiada znacznie większe możliwości. XML Schema jest strukturą XML, w odróżnieniu od DTD nie będącego częścią standardu XML. Dokumenty zawierające definicje XML Schema zapisuje się zwykle w plikach z rozszerzeniem .xsd (od XML Schema Definition).
\\
\\
\textbf{Funkcjonalność DTD może zostać zastąpiona przez XMLSchema. Należy jednak pamiętać, że składnia XMLSchema sama w sobie jest zazwyczaj definiowana przez DTD.}\\
\\
Dokładniejszy opis różnic pomiędzy DTD a XMLSchema oraz ich składnię można podejrzeć tutaj:\\
http://www.sitepoint.com/xml-dtds-xml-schema/
\chapter{Jezyki i~metody programowania}
\PartialToc
%\startcontents[chapters]
%\printcontents[chapters]{}{1}{\section*{\contentsname}}
\section{IT1A\_U05,IT1A\_U03}
\textbf{Wskaż prawdziwe stwierdzenia~o poniższym fragmencie kodu XHTML 1.0 Strict}
\begin{lstlisting}[language=html]
	<p><a  href = http://www.agh.edu.pl><br></p>
\end{lstlisting}

\vspace{0.4cm}
\noindent
\begin{enumerate}
\item Wyrażenie  http://www.agh.edu.pl  powinno zostać ujęte~w cudzysłów.  
\item   Znacznik br powinien byc oznaczony jako jednoczesny znacznik otwierajacy~i zamykający poprzez <br/>
\item Brakuje nazwy hiperłącza (pomiędzy <a></a>).
\item Brakuje znacznika zamykającego dla~a, tuż przed zamknięciem p.
\end{enumerate}


\section{IT1A\_W10,IT1A\_W06, IT1A\_U05, IT1A\_U03, IT1A\_U10}
\textbf{Która~z poniższych metod~w języku JavaScript zwraca element~o unikalnym identyfikatorze form?
 }
\begin{lstlisting}[language=html]
	document.getElementById('form').
\end{lstlisting}

\section{1A\_W10,IT1A\_W06, IT1A\_U05, IT1A\_U03}
\textbf{Co jest efektem działania poniższego programu~w języku PHP.}
\begin{lstlisting}[language=php]
	<?php
	$wiek = array('ala'  => 12, 'ela' => 22, 'franek' => 54);
	foreach ($wiek as $k =>$w)
		echo $k.' '.$w."\n";
	?>
\end{lstlisting}
\vspace{0.4cm}
\noindent
Tablice~w PHP działają jak mapy, mają pary kluczy wraz~z wartościami.~W związku~z tym zostanie wypisany następujący ciąg:

ala  12

ela  22

franek  54
\vspace{0.4cm}

\textbf{Zaznacz prawdziwe stwierdzenia dotyczące poniższego kodu~w języku JavaScript}
\begin{lstlisting}[language=html]
		car = new Array();
		car[0] = new Object();
		car[0].make = 'Fiat';
		car[0].vin = '123';
		car[1] = new Object();
		car[1].make = 'Ford';
		car[1].vin = '456';
		
		for(idx in car){
			for(prop in car [idx]){
				document.write(car[idx][prop]);
			}
		}
\end{lstlisting}

\vspace{0.4cm}
\noindent
Pierwsza pętla operuje po obiektach,~a druga po ich wartościach.~W związku~z tym zostanie wypisane: Fiat123Ford456 (~w miejscu,~w którym został wstawiony kod).
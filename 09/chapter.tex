\chapter{Jezyki i~metody programowania}
\PartialToc
%\startcontents[chapters]
%\printcontents[chapters]{}{1}{\section*{\contentsname}}
\section{IT1A\_W10,IT1A\_U10}
\textbf{Serwery DNS oferują:}
\vspace{0.4cm}
\noindent
\begin{enumerate}
\item Translację nazwy kanonicznej $\rightarrow$ IP
\item Translację odwrotną (Reverse Lookup): IP $\rightarrow$ nazwa
\end{enumerate}
\vspace{0.4cm}
Najważniejsze typy rekordów DNS oraz ich znaczenie:
\begin{description}

\item[A] adres hosta (ipV4) 
\item[AAAA]  adres hosta (ipV6) 
\item[PTR]  	
	rekord wskaźnika - mapuje adres IPv4 lub IPv6 na nazwę kanoniczną hosta. 		Określenie rekordu PTR dla nazwy hosta (ang. hostname) w domenie in-addr.arpa 	(IPv4), bądź ip6.arpa (IPv6), który odpowiada adresowi IP, pozwala na 			implementację odwrotnej translacji adresów DNS (ang. reverse DNS lookup) 
\item[MX] mapuje nazwę domeny DNS na nazwę serwera poczty oraz jego priorytet 
\item[CNAME]	alias nazwy domeny. Wszystkie wpisy DNS oraz poddomeny są 	poprawne także dla aliasu 
\item[TXT]	rekord ten pozwala dołączyć dowolny tekst do rekordu DNS 
\item[NS] Name Server, serwer nazw dla danej strefy 
\item[SOA]    rekord adresu startowego uwierzytelnienia (ang. start of authority record) ustala serwer DNS dostarczający autorytatywne informacje o domenie internetowej 
\end{description}

Narzędzia do odpytywania DSN: host, dig


\section{IT1A\_W10,IT1A\_U10}
\textbf{Zaznacz prawdziwe stwierdzenie. Protokół HTTP w  wersji 1.1}
\vspace{0.4cm}
\noindent

Podobnie jak większość protokołów sieciowych, HTTP uzywa modelu klient-serwer: klient HTTP otwiera połączenie i wysyła komunikat żądania do serwera HTTP. Serwer następnie zwraca komunikat odpowiedzi, zwykle zawierający zasób, o który został poproszony.
Portem zarezerwowanym dla HTTP jest 80 port TCP oraz rzadziej używane porty 8008 i 8080.\\

\textsc{\textbf{Struktura protokołu HTTP}}

\begin{enumerate}
\item Żądanie
	\begin{enumerate}
	\item{polecenie}
	\item{nagłówki (0 lub więcej)}
	\item pusta linia
	\item dane
	\end{enumerate}
\item Odpowiedź
	\begin{enumerate}
		\item pojedyncza linia statusu: \textit{protokół kod opis}
		\item{nagłówki (0 lub więcej)}
		\item pusta linia
		\item dane
	\end{enumerate}
\end{enumerate}


\textsc{\textbf{Kody statusu}}

\begin{description}
\item[1xx] powiadomienie
\item[2xx] powodzenie – 200 OK
\item[3xx] przekierowanie do innego URI – 301 Moved Permanently
\item[4xx] błąd po stronie klienta – 404 Not Found
\item[5xx] błąd po stronie serwera – 500 Server Error
\end{description}

\textsc{\textbf{metody http}}
\begin{description}

\item[OPTIONS] — pozwala klientowi ustalić opcje i/lub wymagania związane z danym zasobem, albo możliwości danego serwera, nie implikując działań zasobu i nie inicjując pobierania zasobu.
\item[GET] — pobiera informacje zidentyfikowane przez URI, do którego zostało zgłoszone żądanie. Jeżeli URI zidentyfikuje proces wytwarzający dane, to wytworzone dane zostaną zwrócone jako jednostka.

\item[HEAD] — identyczny z GET, z tym że serwer nie zwraca w odpowiedzi treści komunikatu. Metoda ta uzyskuje informacje dotyczące jednostki nie przesyłając samej treści jednostki i jest wykorzystywana do testowania ważności, dostępności oraz niedawnych modyfikacji łączy hipertekstowych.

\item[POST] — żąda, aby serwery przyjmowały jednostkę załączoną w żądaniu,  POST jest wykorzystywany do przypisywania zasobów, do wysyłania komunikatów, do przedkładania danych formularzy, itp.

\item[PUT] — żąda, aby jednostka załączona w żądaniu została zapamiętana pod URI, do którego zostało zgłoszone żądanie. Jeżeli dany zasób wspomniany w URI już istnieje, to przesyłaną jednostkę uznaje się za wersję zmodyfikowaną. Jeżeli URI nie wskazuje na istniejący zasób, a żądający użytkownik może zdefiniować URI jako nowy zasób, to zasób jest tworzony na serwerze.

\item[DELETE] — żąda, aby serwer usunął zasób zidentyfikowany przez URI.

\item[TRACE] — wywołuje zdalną pętlę zwrotną komunikatu żądania. Ostateczny odbiorca żądania odbija otrzymany komunikat z powrotem do klienta. TRACE pozwala klientowi zobaczyć co jest odbierane na drugim końcu łańcucha żądania. Informacja ta może być wykorzystywana do testowania, lub znajdywania uszkodzeń.

\item[CONNECT] — nazwa metody zarezerwowana do wykorzystywania wraz z proxy, który może dynamicznie przełączyć się na pełnienie funkcji tunelu, przy użyciu, na przykład, tunelowania z wykorzystaniem warstwy zabezpieczeń łączy (SSL). Proxy to program pośredniczący, pełniący zarówno funkcję serwera, jak i klienta, w celu zgłaszania żądań w imieniu innych klientów.
\end{description}

\textsc{\textbf{nagłówki}}\\
Z wykładu:
\begin{description}

\item[Klient:] \hfill \\
	\begin{description}
	\item[From]: – zwykle email
	\item[User-Agent:] – identyfikacja klienta: Nazwa/Wersja Host: – HTTP 1.1
	\end{description}
\item[Serwer:] \hfill \\
	\begin{description}
		\item[Server:] - analogicznie jak User-Agent
		\item[Last-Modified:] – Data i godzina modyfikacji zasobu 
		\item[Connetion:] – rodzaj połączenia (close dla HTTP 1.0)
		\item[Content-Type:] – tym MIME, np. text/html, application/octet-stream
		\item[Content-Length:] – rozmiar w bajtach
	\end{description}
\end{description}
\href{http://www.wikiwand.com/pl/Lista_nag%C5%82%C3%B3wk%C3%B3w_HTTP}{Wincyj nagłówków: wiki} 
\\


\textsc{\textbf{wybrane różnice między wersją HTTP 1.0 i 1.1}}
\begin{enumerate}
\item{\textbf{ w 1.1 może być więcej niż 1 request}\\
Nagłówek Keep-Alive jest rozszerzeniem HTTP/1.0. \\
W HTTP/1.1 ten nagłówek nie jest potrzebny, gdyż połączenia Keep-Alive są domyślne (zachowanie zmienia Connection: close)}
\item{\textbf{w  1.1 w zapytaniu można określić hosta docelowego}\\
GET / HTTP/1.1 \\
Host: kis.agh.edu.pl}
\item{\textbf{nowa wersja wprowadza nagłówki If-Unmodified-Since, If-Match, If-None-Match}}

\end{enumerate}






\section{IT1A\_W10,IT1A\_U10}
\textbf{Do bezpośredniej komunikacji z serwerem WWW służą następujące narzędzia}
\vspace{0.4cm}
\noindent
 \begin{enumerate}
 \item{nc (netcat)} 
 \item{każdy klient protokołu zdalnego logowania (ssh, telnet) ?}
 \end{enumerate}
 
 \section{IT1A\_U05,IT1A\_U03}
 \textbf{Dany jest poniższy fragment kodu XHTML 1.0 Strict}
\vspace{0.4cm}
\noindent
\begin{lstlisting}[language=html]
	<img src="http://www.agh.edu.pl/i.jpg" 
			width ="320"
			height ="240" 
			alt="logo AGH" />
\end{lstlisting}
\textbf{Obrazek \textit{i.jpg} ma rozmiary 1024x768. Zaznacz prawdziwe stwierdzenia}
\\\\
Obrazek zostanie przeskalowany do podanych rozmiarów, zniekształcając go jeśli proporcje nowych wymiarów nie są takie same. \\
Domyślą jdenostką artybutów \textit{width} i \textit{height} są piksele. Można też podać wartość \%, odnosi się ona wtedy do ilości wolnego miejsca jakim dysponuje element.\\
Jeśli podane są wartości \textit{width} i \textit{height} , przeglądarka rezerwuje odpowiednie miejsce podczas layout'owania (po załadowaniu obrazka strona się nie przesunie).\\
W miejscu niezaładowanego obrazka będzie tekst podany pod atrybutem \textit{alt}